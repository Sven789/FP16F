\documentclass[a4paper,11pt,DIV=11]{scrartcl}
\usepackage[utf8]{inputenc}
\usepackage[version=4]{mhchem}
\usepackage[ngerman]{babel}
\usepackage{amsfonts, amsmath, amssymb}
\usepackage{graphicx}
\usepackage{float}
\parindent0pt

\title{Transmissions-Elektronen-Mikroskopie}
\author{Adrian Messow, Sven Mehrkens \\
Tutor: ???}
\date{Durchführung: 14.12.2017 \\ Abgabe: ??? }

\begin{document}
\maketitle
\section{Einführung}
In diesem Versuch werden die Grundlagen des Transmissionselektronenmikroskops (TEM) vermittelt. Dabei werden nach einer grundlegenden Justierung die Beugungsbilder im Hellfeld und Dunkelfeld an einer GaAs-Probe aufgenommen und indiziert. Durch Aufnahmen von InGaAs-Quantentrögen wird anschließend einerseits über die Intensitätsverteilung und andererseits über die Gitterkonstanten die Indiumkonzentration bestimmt.

\section{Theoretische Grundlagen}
	
\section{Versuchsaufbau und Versuchsdurchführung}

\section{Auswertung}

\section{Zusammenfassung}

\section*{Anhang}

\end{document}